
Chapter I How did Marx Invent the Symptom
Marx, Freud: The Analysis of form
- Une thèse lacanienne: "According to Lacan, It was none other than Karl Marx who invented the notion of symptom"
Dissolution!, le 18 mars 1980 ,"J’ai rendu hommage à Marx comme à l’inventeur du symptôme"
Lacan le dit dans la séance du séminaire R.S.I. du 18 février 1975 : « Le symptôme n’est pas définissable autrement que par la façon dont chacun jouit de l’inconscient en tant que l’inconscient le détermine. Cherchez l’origine de la notion de symptôme, qui n’est pas du tout à chercher dans Hippocrate, qui est à chercher dans Marx.»

- La thèse selon laquelle il existe une "fundamental homology between the ... analysis of commidity and of dreams" via the avoidence, of x in DieWarenform and respectively of DerLantenterTrauminhalt in DieTraumarbeit.
A reproach "based on a fundamental theoretical error:" La contradiction de Hans-Jürgen Eysenck: The majority of Freud's dreams examples (including the famous Irma's Injektion) are not of a sexual nature. Id est dieLantenterTruaminhalt is the place of DerUnbewussteWunsch. But this in an error. Via DerPrimärvorgung the most ordinary bewusste-thoughts can be translated into DasUnbewusste. But it is Die Wunschen that can not be articulated that is of a sexual nature, DieUrverdrängung. 
The real 'Kernel'(DerUnbewussteWunsch) is parcellated out into the dream into oridinary thought, picked up by Eysenck criticism, but we are to look to DieTraumarbeit to piece togeather this Kernel.
In Marx's critic of DieWarenform: unmasking Supply&Demand to be a mere accidentality, as supplement with his LabourTheoryofValue. But this is not sufficient for understanding DieWarenform. Its true kernel is likewise in process of value creation itself.


Dealing: DieTraumdeutung/InterpretationOfDreams DerManifesterTrauminhalt DerLantenterTrauminhalt DerTrauminhalt/DreamContent DieTraumarbeit/Dreamwork DieLibido/pansexualism DerUnbewussteWunsch/UnconsciousWishDesire DerPrimärvorgung DasUnbewusste
 DieWarenform/CommodityForm 


The unconscious of commodity-form
"The Marxian analysis of the commodity-form ... offers a kind of matrix enabling us to generate all other forms of the 'fetishistic inversion'"
"In the commodity-form there is definately more at stake than the commiditiy-form itself". "In the structure of commidity-form it is possible to find the transcendental subject: ... the Kantian transcendental subject"

In Kant's 'Kritik der reinen Vernunft (1787 2nd Edition Preface). Kant's Copernicueran revolution
7+5 = 12 is eine synthetische Urteil a priori. Thus Kant concludes that all pure mathematics is synthetic though a priori. Phenomena (appearances) are structured by a priori conditions of the human mind. These conditions fall into two main faculties Die formen der Anschauung: Der Raum und Die Zeit. Die zwölf Kategorien des Verstandes is the frabric of how the transcendental subject, das ich denke, structure phenomina. 

Alfred Sohn-Rethel, Intellectual and Manual Labour (1978), 
Das reale Abstraktion: "the act of abstraction is at work in the very effective process of the exchange of commodities"
Als Ob and its simmilarity with the formula for Fetishistic Disavowel "Je sais bien .... Mais".
Eg The Materiality of Money

Dealing: DasUrteil/Verdict DasSynthetischeUrteilAPriori DieTranzendentalphilosophie/TranscendentalIdealism
